% !TeX spellcheck = de_DE
\documentclass[DIV=15, fontsize=11pt]{scrartcl}
\setkomafont{sectioning}{\normalfont\bfseries\boldmath}

\usepackage[pass]{geometry}

\usepackage[british, main=ngerman]{babel}
\usepackage{xcolor}
\usepackage{tikz}


\usepackage{enumitem}
\setlist{nolistsep}

\setlength\parindent{0pt}
\setlength\parfillskip{1em plus 1fil}
\setlength\parskip{0pt}

%\usepackage{fontspec}
%\setmainfont{Linux Libertine O}
%\setsansfont{Linux Biolinum O}
\newcommand\hugefont[1]{\raisebox{-5pt}{\fontsize{50}{60}\sffamily\bfseries\selectfont #1}}
\newcommand\hugemathfont[1]{{\fontsize{40}{48}\sffamily\bfseries\selectfont #1}}
\newcommand\biggerfont[1]{{\fontsize{30}{36}\sffamily\bfseries\selectfont #1}}
\newcommand\bigfont[1]{{\fontsize{20}{24}\sffamily\bfseries\selectfont #1}}
\newcommand\subbigfont[1]{{\fontsize{16}{20}\sffamily\bfseries\selectfont #1}}
%\newcommand\textfontsize[1]{\fontsize{10}{12}\sffamily\selectfont #1}
\newcommand\textfontsize[1]{\fontsize{9}{10}\sffamily\selectfont #1}

\newcommand\rede[1]{\glqq #1\grqq{}}
\newcommand\textrule[1]{\textbf{#1}}
\newcommand\Item{$\bullet$\kern0.333em}

\newdimen\cardwidth
\cardwidth=60mm
\newdimen\cardheight
\cardheight=90mm
\newdimen\cardborder
\cardborder=5mm
\newdimen\cardborderradius
\cardborderradius=6mm
\newdimen\cardtextwidth
\cardtextwidth=\cardwidth
\advance\cardtextwidth by -2.75\cardborder
\newdimen\cardlinewidth
\cardlinewidth=0.4pt

\newcount\cardnum
\cardnum=0
\newcount\auxnum
\newcommand\CardBack[1][bgcolor1]{%
  \advance\cardnum by 1%
  \begin{tikzpicture}
%	\draw[line width=\cardlinewidth, rounded corners=5mm] (0,0) rectangle (\cardwidth,\cardheight) coordinate (URex);
	\path (0,0)--(\cardwidth,\cardheight) coordinate[midway] (Center) coordinate (URex);
	\path (URex)--+(-\cardborder,-\cardborder) coordinate (UR);
	\fill[#1, rounded corners=\cardborderradius] (\cardborder,\cardborder) coordinate (LL) rectangle (UR);
    \node at (Center) {\hugefont{git}};
  \end{tikzpicture}%
  \ifnum\cardnum>2
  	\cardnum=0
  
  	\vskip-1pt
  \fi
}
\newenvironment{CardFront}[1]{%
  \global\advance\cardnum by 1%
  \begin{tikzpicture}
  	% Schnittrahmen
    \draw[line width=\cardlinewidth, rounded corners=\cardborderradius, fill=white] (0,0) rectangle (\cardwidth,\cardheight) coordinate (URex);

    \path (0,0)--(\cardwidth,\cardheight) coordinate[midway] (Center);
    \path (URex)--+(-\cardborder,-\cardborder) coordinate[midway] (URmed) coordinate (UR);
    \coordinate (ULex) at (0,\cardheight);
    \coordinate (LRex) at (\cardwidth,0);
    
    % innerer Rahmen
    \fill[bgcolor, rounded corners=\cardborderradius] (\cardborder,\cardborder) coordinate (LL) rectangle (UR);
    
    \path (LL)-|(UR) coordinate[midway] (LR);
    \path (LL)|-(UR) coordinate[midway] (UL);
    \path (UL)--(ULex) coordinate[midway] (ULmed);
    \path (LL)-|(UR) coordinate[midway] (LR);
    \path (LL)--(0,0) coordinate[midway] (LLmed);
    \path (LR)--(LRex) coordinate[midway] (LRmed);

    \path (Center)+(0,0.075\cardheight) coordinate (UCenter);
	
	% Überschrift
	\path (UCenter)+(0,0.25\cardheight) node {\bigfont{#1}};
}{
  \end{tikzpicture}%
  \ifnum\cardnum>2
	\global\cardnum=0

	\vskip-1pt
  \fi
}

\colorlet{bgcolor}{gray!20}
\colorlet{bgcolor1}{green!20}
\colorlet{bgcolor2}{orange!25} % niedrige CRs
\colorlet{bgcolor3}{red!20}% hohe CRs
\colorlet{bg-green}{green!65}
\colorlet{bg-yellow}{yellow!75}
\definecolor{bg-blue}{rgb}{0.25,0.55,1}
\colorlet{paleblue}{cyan!35}
\colorlet{errorcolor}{red!65}
\colorlet{warncolor}{orange!75}
\colorlet{nullcolor}{bg-yellow}
\colorlet{middlecolor}{bg-green}
\colorlet{highcolor}{bg-blue}
\colorlet{oddcolor}{magenta!65}

\newcommand\CardTextField[1]{% Text
    \path (UCenter)+(0,-0.25\cardwidth-0.015\cardheight) node[anchor=north] {%
	\begin{minipage}{\cardtextwidth}
		\textfontsize\rightskip0pt minus 0.25em
		#1
	\end{minipage}};
}
\newcommand\CardFrontTikzBase[3]{% Überschrift, Tikz-Graphik, Text
	\begin{CardFront}{#1}
		% Tikz-Graphik
		#2
		
		% Textbereich
		\CardTextField{#3}
	\end{CardFront}%
}
\newcommand\CardFrontTikzCircle[4]{% Überschrift, Tikz-Graphik, Farbe, Text
	\CardFrontTikzBase{#1}{%
%		\draw[line width=1.5pt] (UCenter)+(-0.45\cardwidth,0)--+(0.45\cardwidth,0);
		\draw[line width=1.5pt, fill=#3] (UCenter) circle[radius=0.25*\cardwidth];
		#2}{#4}%
}
\newcommand\CardFrontTikzBox[5][0.25\cardwidth]{% halbe Breite, Überschrift, Tikz-Graphik, Farbe, Text
	\CardFrontTikzBase{#2}{%
		\draw[line width=1.5pt, rounded corners, fill=#4] (UCenter)++(-#1,-0.2\cardwidth) rectangle ++(2*#1,0.4\cardwidth);
		#3}{#5}%
}
\newcommand\CalcNumberColor[1]{% Zahlwert
	\ifnum#1<0
		\colorlet{numcolor}{errorcolor}%
	\else%
		\ifnum#1<8%
			\auxnum=#1
			\ifnum#1=0\else\advance\auxnum by -1\fi%
			\multiply\auxnum by 14
			\colorlet{numcolor}{middlecolor!\the\auxnum!nullcolor}%
		\else%
			\ifnum#1<32
				\auxnum=#1
				\advance\auxnum by -8
				\multiply\auxnum by 42
				\divide\auxnum by 10
				\colorlet{numcolor}{highcolor!\the\auxnum!middlecolor}%
			\else%
				\colorlet{numcolor}{highcolor}%
			\fi%
		\fi%
	\fi%
%
	\auxnum=#1
	\ifnum\auxnum<0\else%
		\ifnum\auxnum<16
			\def\sixteen{white}%
		\else%
			\def\sixteen{black}%
			\advance\auxnum by -16
		\fi%
		\ifnum\auxnum<8
			\def\eight{white}%
		\else%
			\def\eight{black}%
			\advance\auxnum by -8
		\fi%
		\ifnum\auxnum<4
			\def\four{white}%
		\else%
			\def\four{black}%
			\advance\auxnum by -4
		\fi%
		\ifnum\auxnum<2
			\def\two{white}%
		\else%
			\def\two{black}%
			\advance\auxnum by -2
		\fi%
		\ifnum\auxnum<1%
			\def\one{white}%
		\else%
			\def\one{black}%
		\fi%
	\fi%
	\auxnum=#1
}
\newcommand\DrawNumber[1][]{
	\ifnum\auxnum<0
		\path (UCenter)+(0,0.025\cardwidth) node[anchor=center] {\hugefont{X}};
	\else
		\path (UCenter)+(0,0.025\cardwidth) node[anchor=center] {#1\hugefont{\number\auxnum}};
		% Binärcode
		\ifnum\auxnum<16
			\path (UCenter)++(-0.09*\cardwidth,-0.125*\cardwidth) coordinate (aux);
		\else
			\path (UCenter)++(-0.12*\cardwidth,-0.125*\cardwidth) coordinate (aux);
			\draw[fill=\sixteen] (aux) circle[radius=1mm];
			\path (aux)++(0.06*\cardwidth,0) coordinate (aux);
		\fi
		\draw[fill=\eight] (aux) circle[radius=1mm];
		\path (aux)++(0.06*\cardwidth,0) coordinate (aux);
		\draw[fill=\four] (aux) circle[radius=1mm];
		\path (aux)++(0.06*\cardwidth,0) coordinate (aux);
		\draw[fill=\two] (aux) circle[radius=1mm];
		\path (aux)++(0.06*\cardwidth,0) coordinate (aux);
		\draw[fill=\one] (aux) circle[radius=1mm];
	\fi
}
\newcommand\DrawNumberBorder{%
	\ifnum\auxnum<0\else%
		\node[draw, line width=1pt, fill=numcolor, circle, inner sep=0pt, anchor=north west] (ULborder) at (ULmed) {\hphantom{\bigfont{0}}\clap{\bigfont{\number\auxnum}}\hphantom{\bigfont{0}}};%
		\node[draw, line width=1pt, fill=numcolor, circle, inner sep=0pt, anchor=north east] (URborder) at (URmed) {\hphantom{\bigfont{0}}\clap{\bigfont{\number\auxnum}}\hphantom{\bigfont{0}}};%
%		\node[draw, line width=1pt, fill=numcolor, circle, inner sep=0pt, anchor=north east, rotate=180] at (LLmed) {\hphantom{\bigfont{0}}\clap{\bigfont{\number\auxnum}}\hphantom{\bigfont{0}}};%
%		\node[draw, line width=1pt, fill=numcolor, circle, inner sep=0pt, anchor=north west, rotate=180] at (LRmed) {\hphantom{\bigfont{0}}\clap{\bigfont{\number\auxnum}}\hphantom{\bigfont{0}}};%
	\fi
}
\newcommand\DrawSignAtBorder[2][numcolor]{% Farbe, Symbol
	\node[draw, line width=1pt, fill=#1, rectangle, rounded corners, inner sep=1pt, minimum height=20pt, anchor=north west] at (ULmed) {\hphantom{\bigfont{0}}\clap{\bigfont{#2}}\hphantom{\bigfont{0}}};%
	\node[draw, line width=1pt, fill=#1, rectangle, rounded corners, inner sep=1pt, minimum height=20pt, anchor=north east] at (URmed) {\hphantom{\bigfont{0}}\clap{\bigfont{#2}}\hphantom{\bigfont{0}}};%
%	\node[draw, line width=1pt, fill=#1, rectangle, rounded corners, inner sep=1pt, minimum height=20pt, anchor=north east, rotate=180] at (LLmed) {\hphantom{\bigfont{0}}\clap{\bigfont{#2}}\hphantom{\bigfont{0}}};%
%	\node[draw, line width=1pt, fill=#1, rectangle, rounded corners, inner sep=1pt, minimum height=20pt, anchor=north west, rotate=180] at (LRmed) {\hphantom{\bigfont{0}}\clap{\bigfont{#2}}\hphantom{\bigfont{0}}};%
}
\newcommand\DrawNumberAtBorder[1][numcolor]{% Farbe
	\ifnum\auxnum<0\else%
		\DrawSignAtBorder[#1]{\number\auxnum}
	\fi
}
\newcommand\CardFrontCommit[2][\centering\rede{git push!}]{% Text, Zahlwert
	\CalcNumberColor{#2}%
	\CardFrontTikzCircle{commit}{\DrawNumber\DrawNumberBorder}{numcolor}{#1}%
}
\newcommand\CardFrontCommitPlus[4][\centering\rede{git push!}]{% Text, Zahlwert, Symbol, Farbe
	\CalcNumberColor{#2}%
	\CardFrontTikzCircle{commit}{\DrawNumber\DrawNumberBorder %
%		\draw[line width=1.25pt, rounded corners, fill=#4] (UCenter)++(30:0.275\cardwidth) coordinate (c)++(-90:0.0866\cardwidth)--++(0:0.15\cardwidth)--++(120:0.3\cardwidth)--++(-120:0.3\cardwidth)--cycle;
%		\node[anchor=center] at (c) {\biggerfont{#3}};
		\draw[line width=1pt, rounded corners, fill=#4] (ULborder.south)++(0,-0.035\cardheight) coordinate (c)++(-90:0.0433\cardwidth)--++(0:0.075\cardwidth)--++(120:0.15\cardwidth)--++(-120:0.15\cardwidth)--cycle;
		\node[anchor=center] at (c) {\subbigfont{#3}};
		
		\draw[line width=1pt, rounded corners, fill=#4] (URborder.south)++(0,-0.035\cardheight) coordinate (c)++(-90:0.0433\cardwidth)--++(0:0.075\cardwidth)--++(120:0.15\cardwidth)--++(-120:0.15\cardwidth)--cycle;
		\node[anchor=center] at (c) {\subbigfont{#3}};
	}{numcolor}{#1}%
}
\newcommand\CardFrontSign[4]{% Überschrift, Zeichen, Farbe, Text
	\CardFrontTikzBox{#1}{%
		% Mitte
		\path (UCenter) node[anchor=center] {\hugefont{#2}};
		% Ecken: TODO
		\DrawSignAtBorder[#3]{#2}
	}{#3}{#4}%
}

\newcommand\tagnode[3][line width=1pt]{% scale, position
	\draw[rounded corners=0.025*#2, fill=warncolor, #1] (#3)++(-0.2*#2,-0.175*#2)--++(0.4*#2,0)--++(0.175*#2,0.175*#2)--++(-0.175*#2,0.175*#2)--++(-0.4*#2,0)--cycle;
	\draw[#1, fill=bgcolor] (#3)+(0.235*#2,0) circle[radius=0.05*#2];
}
\newcommand\CardRelease{%
	\CardFrontTikzBase{release}{%
		% Mitte
		\path (UCenter)++(-0.075\cardwidth,0) coordinate (tagcenter);
		\tagnode[line width=1.5pt]{\cardwidth}{tagcenter}
		\node at (tagcenter) {\bigfont{v\,1.x}};
		% Ecken
		\path (UL)+(0,-0.05\cardwidth) coordinate (tagcenter);
		\begin{scope}[rotate=90]
			\tagnode{0.25*\cardwidth}{tagcenter}
		\end{scope}
		\path (UR)+(0,-0.05\cardwidth) coordinate (tagcenter);
		\begin{scope}[rotate=90]
			\tagnode{0.25*\cardwidth}{tagcenter}
		\end{scope}
	}{%
	\centerline{\rede{git tag .\,.\,. und Lieferung!}}\vskip0.5ex
%	Ich beschließe, dass der aktuelle Stand ausgeliefert wird:
	\Item Falls das Ziel erreicht ist, wertet alle Karten im Repo.\ aus und legt sie sowie die aktuellen Issues und alle Handkarten auf den Abwurfstapel.\\
	\Item Leg diese Karte auf den Abwurfstapel und ziehe einen neuen CR.}%
}
\newcommand\CardRevert{%
	\CardFrontTikzBox{revert}{%
		% Mitte
		\begin{scope}[line width=2pt]
			\path (UCenter)--+(0.05\cardwidth,0.075\cardheight) coordinate (aux);
			\node[draw, fill=gray, circle, radius=1.5mm]  (A) at (aux) {};
			\path (UCenter)--+(0.05\cardwidth,-0.075\cardheight) coordinate (aux);
			\node[draw, fill=gray, circle, radius=1.5mm]  (B) at (aux) {};
			\draw (A)--(B);
			\draw[->] (A)..controls +(-0.2\cardwidth,0) and +(-0.2\cardwidth,0)..(B);
		\end{scope}
		% Ecken: TODO
	}{warncolor}{%
		\centerline{\rede{git revert!}}\vskip1ex
		Der gerade geschriebene Code ist fehlerhaft und muss wieder weg:\vskip0.5ex
		\Item Muss unmittelbar auf einen Commit folgen und negiert ihn.\\
		\Item Leg diese Karte und den Commit auf den Abwurfstapel.}%
}
\newcommand\CardMergeConflict[1]{% Zahlwert
	\CardFrontCommitPlus[%
		\centerline{\rede{git push -- merge conflict!}}\vskip1ex
		Mit diesem Commit verursachst und löst du einen Merge Conflict:\vskip0.5ex
		\Item Übernimm den vorherigen Commit als deinen eigenen. Schieb ihn unter diese Karte, diese beiden Commits gelten ab jetzt als einer.]{#1}{!}{bg-yellow}%
}
\newcommand\CardMergeRequest[1]{% Zahlwert
	\CardFrontCommitPlus[%
		\centerline{\rede{git push -- merge request!}}\vskip1ex
		Mit diesem Commit \textit{bittest} du deinen Kollegen um ein Code Review, dass er natürlich auch durchführt:\vskip0.5ex
		\Item Der nächsten Spieler muss ein Revert ausspielen oder einmal aussetzen.]{#1}{?}{paleblue}%
}
\newcommand\CardIssue[2][CR]{% Zusatz, Zahlwert
	\CalcNumberColor{#2}%
	\CardFrontTikzBox{Issue: #1}{\DrawNumber\DrawNumberAtBorder}{numcolor}{%
		\Item Für diese Lieferung müssen mindestens #2~Codezeilen geschrieben werden.\\
		\auxnum=#2 %
		\divide\auxnum by 2 %
		\Item Bei der Lieferung erhält der Auslieferer \number\auxnum~Zusatzpunkte.}%
}
\newcommand\CardBug[2][0.25\cardwidth]{% halbe Breite, Zahlwert
	\auxnum=#2
	\CardFrontTikzBox[#1]{Issue: Bug}{%
%		\node at (UCenter) {\hugemathfont{+}\!\hugefont{#2}};
		\DrawNumber[\hugemathfont{+}\!]
		\DrawNumberAtBorder[errorcolor]
		}{errorcolor}{%
		\rede{Ich habe einen Fehler gefunden, der auch noch behoben werden muss.}\vskip0.5ex
		\Item Leg die Karte zum aktuellen CR.\\
		\Item Für die aktuelle Lieferung müssen noch #2~Codezeilen zusätzlich geschrieben werden.}%
}

\title{Git -- das Kartenspiel}
\author{Philipp Müller}

\begin{document}

\maketitle

Die Spieler bilden ein Team agiler Software-Entwicklung. Ziel des Spiels ist es, möglichst viele der eigenen Codezeilen durchzubringen und die anderen Spieler dabei zu behindern.



\section{Aufbau}\fboxsep=0pt
Die Karten mit der \colorbox{bgcolor1}{grünliche Rückseite\strut} werden gut gemischt, jeder Spieler erhält drei davon auf die Hand, der Rest bildet den Zugstapel.

Die Karten mit der \colorbox{bgcolor2}{rötlichen Rückseite\strut} werden auch gut gemischt und bilden den Issue-Stapel. Eine Issue-Karte wird gezogen und aufgedeckt. Sie bildet das Ziel der ersten Spielrunde.

%\CardFrontCommit{2}


\section{Ablauf}
Das Spiel wird in Runden gespielt, die jeweils der Entwicklung einer Version der Software entsprechen. Diese beginnen mit dem Aufdecken einer Issue-Karte, die das Ziel der Runde angibt, und Enden mit einer Lieferung. Dazwischen spielen alle Spieler reihum immer eine Handkarte aus und sagen dabei das, was auf der Karte in Anführungszeichen~(\rede{.\,.\,.}) steht.

Es gibt drei Arten von Handkarten:
\begin{itemize}
	\item \textrule{Commit}-Karten werden am Remote Repository angelegt und müssen dann mit einem Marker markiert werden, damit man später weiß, wer welche Punkte bekommt. Ihr Zahlwert entspricht der Anzahl Codezeilen.
	\item \textrule{Release}-Karten beenden die aktuelle Runde, die dann \textrule{ausgewertet} wird~(s.\,u.). Sie können nur ausgespielt werden, wenn das aktuelle Ziel erreicht ist.
	\item Sonstige Karten werden entsprechend ihrer Aufschrift ausgewertet und landen danach meistens auf dem Abwurfstapel.
\end{itemize}

Bestimmte Karten dürfen nur unter bestimmen Bedingungen ausgespielt werden. Falls jemand keine seiner Handkarten ausspielen kann, legt er sie alle auf den Abwurfstapel.

Am Ende des Zuges \textrule{stockt} man seine Handkarten wieder auf~(s.\,u.) und dann ist der nächste Spieler am Zug.

Die Spielrunde ist beendet, sobald jemand eine Release-Karte ausspielt. Dann wird die aktuelle Runde \textrule{ausgewertet}~(s.\,u.) und danach beginnt die nächste Spielrunde mit dem Aufdecken der nächsten Issue-Karte oder das Spiel ist ganz beendet~(s.\,u.).


\subsection{Aufstocken}
Falls man am Ende seines Zuges weniger als drei Handkarten hat, zieht man neue Karten vom Zugstapel bis man wieder drei auf der Hand hat.


\subsection{Auswerten}
Sobald jemand eine Release-Karte ausspielt, wird eine neue Version der Software geliefert. D.\,h.\ die aktuelle Runde wird ausgewertet. Jeder Spieler erhält Punkte entsprechend der Codezeilen all seiner Commits im Remote Repository zwischen dem CR und der Lieferung. Der Spieler der Release-Karte erhält zusätzliche Punkte wie auf der Issue-Karte angegeben.

TODO: Beispiel für eine Auswertung



\section{Spielende}
Am Anfang sollten alle Teilnehmer festlegen, wann das Spiel beendet ist (Definition of Done), z.\,B. nach 5~Lieferungen oder sobald ein Spiel mindestens 100~Punkte erreicht hat.

Gewonnen hat dann der Spieler mit der höchsten Punktezahl.



\vfill
Es folgen die Seiten mit den Karten zum Ausdrucken.
\newdimen\auxdim
\auxdim=3\cardwidth
\advance\auxdim by 3\cardlinewidth
\newgeometry{top=10mm,bottom=10mm,textwidth=\auxdim}
\cardnum=0

\CardFrontCommit{1}%
\CardFrontCommit{2}%
\CardFrontCommit{3}%
\CardFrontCommit{4}%
\CardFrontCommit{5}%
\CardFrontCommit{6}%
\CardFrontCommit{7}%
\CardFrontCommit{8}%
\CardFrontCommit{9}%

%\newgeometry{top=10mm,bottom=10mm,textwidth=3\cardwidth}
\CardBack%
\CardBack%
\CardBack%
\CardBack%
\CardBack%
\CardBack%
\CardBack%
\CardBack%
\CardBack%



%
\CardFrontCommit{10}%
\CardFrontCommit{12}%
\CardFrontCommitPlus[%
\centerline{\rede{git push -- refactoring!}}\vskip1ex
	%todo%Mit diesem Commit verursachst und löst du einen Merge Conflict:\vskip0.5ex
	\Item Übernimm die vorherigen drei Commits als deine eigenen. Schieb sie unter diese Karte, alle diese Commits gelten ab jetzt als einer.]{0}{!}{errorcolor}%
\CardMergeConflict{2}%
\CardMergeConflict{3}%
\CardMergeConflict{4}%
\CardMergeRequest{2}%
\CardMergeRequest{3}%
\CardMergeRequest{4}%

%
\CardBack%
\CardBack%
\CardBack%
\CardBack%
\CardBack%
\CardBack%
\CardBack%
\CardBack%
\CardBack%



\CardRevert%
\CardRevert%
\CardRevert%
\CardFrontSign{help}{?}{bg-green}{%
	\centerline{\rede{git help!}}\vskip1ex
	Du weißt gerade nicht weiter und brauchst Hilfe:\vskip0.5ex
	\Item Leg diese Karte auf den Abwurfstapel.\\
	\Item Zieh drei Handkarten zusätzlich, nachdem du \textrule{aufgestockt} hast.}%
\CardFrontSign{status}{?}{bg-yellow}{%
	\centerline{\rede{git status!}}\vskip1ex
	Du hast gerade keine Idee und brauchst erstmal einen Überblick:\vskip0.5ex
	\Item Leg diese Karte auf den Abwurfstapel.\\
	\Item Zieh eine Handkarte zusätzlich, nachdem du \textrule{aufgestockt} hast.}%
\CardFrontTikzBox{stash}{%
	\begin{scope}[line width=2pt]
		\draw[fill=gray] (UCenter)+(-0.15\cardwidth,0) rectangle +(0.15\cardwidth,-0.075\cardheight);
		\draw[-latex, line width=3pt] (UCenter)+(0,0.1\cardheight)--(UCenter);
	\end{scope}
	}{oddcolor}{%
	\centerline{\rede{git stash!}}\vskip1ex
	\Item Leg bis zu zwei Handkarten vor Dir ab.
	\Item Diese können weiterhin wie Handkarten ausgespielt werden, zählen aber nicht beim \textrule{Austocken} mit und gehen bei einem Release nicht verloren.}%
\CardFrontTikzBase{cherry-pick}{%
		% Stiel
		\draw[line width=3pt,green!50!black] (Center)..controls+(0,0.3\cardwidth)..+(0.2\cardwidth,0.35\cardwidth) coordinate[pos=0.8] (aux);
		% Blatt
%		\fill (aux)--+()
		% Frucht
		\shade[ball color=red!80!blue] (Center) circle[radius=0.1\cardwidth];
%		\fill[red!80!blue] (Center) circle[radius=0.1\cardwidth];
	}{%
	\centerline{\rede{git cherry-pick!}}\vskip1ex
	\Item \textrule{Stocke} zuerst deine Handkarten auf.\\
	\Item Wähle dir danach bei einem beliebigen Spieler aus seinen Handkarten eine aus und füge sie deinen hinzu oder spiele sie sofort aus.}%
\CardFrontTikzBox{}{}{white}{todo}%
\CardFrontTikzBox{}{}{white}{todo}%

\CardBack%
\CardBack%
\CardBack%
\CardBack%
\CardBack%
\CardBack%
\CardBack%
\CardBack%
\CardBack%



\CardRelease%
\CardRelease%
\CardRelease%
\CardRelease%
\CardRelease%
\CardBug{5}%
\CardBug{7}%
\CardBug{8}%
\CardBug[0.275\cardwidth]{10}%

\CardBack%
\CardBack%
\CardBack%
\CardBack%
\CardBack%
\CardBack%
\CardBack%
\CardBack%
\CardBack%



\CardIssue{10}%
\CardIssue{12}%
\CardIssue{15}%
\CardIssue{16}%
\CardIssue{20}%
\CardIssue{24}%
\CardIssue{26}%
\CardIssue{30}%
\CardIssue{32}%

\CardBack[bgcolor2]%
\CardBack[bgcolor2]%
\CardBack[bgcolor2]%
\CardBack[bgcolor2]%
\CardBack[bgcolor2]%
\CardBack[bgcolor2]%
\CardBack[bgcolor2]%[bgcolor3]%
\CardBack[bgcolor2]%[bgcolor3]%
\CardBack[bgcolor2]%[bgcolor3]%

\end{document}
