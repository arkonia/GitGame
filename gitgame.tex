% !TeX spellcheck = de_DE
\documentclass[DIV=15, fontsize=11pt]{gitgame}
\renewcommand*{\sectionformat}{\thesection~--~}
\renewcommand*{\subsectionformat}{\thesubsection~--~}

\usepackage[pass]{geometry}

\usepackage[provide=*, british, main=ngerman]{babel}
\usepackage[hidelinks]{hyperref}



\title{Git -- das Kartenspiel\footnote{Dies ist Version~1.0. Die aktuelle Version ist zu finden unter: \url{https://github.com/arkonia/GitGame}}}
\author{Philipp Müller}

\begin{document}

\maketitle

Dieses Spiel parodiert die agile Software-Entwicklung. Die Spieler bilden dabei ein Team von Entwicklern, die gemeinsam eine Software weiterentwickeln. Ziel des Spiels ist es jedoch, möglichst viele der eigenen Codezeilen durchzubringen und die anderen Spieler genau daran zu behindern.



\section{Aufbau}\fboxsep=0pt
Die Karten mit der \bgcolorbox{bgcolor1}{grünliche Rückseite} werden gut gemischt und bildet den Zugstapel. Jeder Spieler erhält drei davon auf die Hand, die sog.\ Handkarten. Diese entsprechen dem lokalen Git-\hskip0ptArbeitsverzeichnis des Spielers.

Die Karten mit der \bgcolorbox{bgcolor2}{rötlichen Rückseite} werden auch gut gemischt und bilden den Issue-Stapel. Eine Issue-Karte wird gezogen und aufgedeckt. Sie bildet das Ziel der ersten Spielrunde.



\section{Ablauf}
Das Spiel wird in Runden gespielt, die jeweils der Entwicklung einer Version der Software entsprechen. Sie beginnen mit dem Aufdecken einer \bgcolorbox{bgcolor2}{Issue-Karte}, die das Ziel der Runde angibt, und Enden mit einer Lieferung. Dazwischen spielen alle Spieler reihum immer genau eine Handkarte aus und sagen dabei das, was auf der Karte in Anführungszeichen~(\rede{.\,.\,.}) steht.

Es gibt drei Arten von \bgcolorbox{bgcolor1}{Handkarten}:
\begin{itemize}
	\item \textrule{Commit}-Karten werden ausgelegt und müssen dann mit einem Spielermarker markiert werden, damit man später weiß, wer welche Punkte bekommt. Ihr Zahlwert entspricht der Anzahl Codezeilen. Die Reihe der ausgelegten Commit-Karten entsprechen dem Remote Repository von Git.
	\item \textrule{Release}-Karten beenden die aktuelle Runde, die dann \textrule{ausgewertet} wird~(s.\,u.~\ref{auswerten}). Sie können nur ausgespielt werden, wenn das aktuelle Ziel erreicht ist, d.\,h.\ wenn die Summe der ausgelegten Zahlwerte dem Wert der Issue-Karte (incl.\ Bugs) entspricht oder diesen übertrifft.
	\item Sonstige Karten werden entsprechend ihrer Aufschrift ausgewertet und landen danach meistens auf dem Abwurfstapel.
\end{itemize}

Bestimmte Karten dürfen nur unter bestimmen Bedingungen ausgespielt werden. Falls jemand an die Reihe kommt und keine seiner Handkarten ausspielen kann, legt er sie alle auf den Abwurfstapel. Er spiel in dieser Runde also keine Karte aus.

Am Ende des Zuges \textrule{stockt} man seine Handkarten wieder auf~(s.\,u.~\ref{aufstocken}) und dann ist der nächste Spieler an der Reihe.

Die Spielrunde ist beendet, sobald jemand eine Release-Karte ausspielt. Dann wird die aktuelle Runde \textrule{ausgewertet}~(s.\,u.~\ref{auswerten}) und danach beginnt die nächste Spielrunde mit dem Aufdecken der nächsten Issue-Karte oder das Spiel ist ganz beendet~(s.\,u.~\ref{ende}).


\subsection{Aufstocken\label{aufstocken}}
Falls man am Ende seines Zuges weniger als drei Handkarten hat, zieht man neue Karten vom Zugstapel bis man wieder drei auf der Hand hat.


\subsection{Auswerten\label{auswerten}}
Sobald jemand eine Release-Karte ausspielt, wird eine neue Version der Software geliefert. D.\,h.\ die aktuelle Runde wird ausgewertet. Jeder Spieler erhält Punkte entsprechend der Codezeilen all seiner Commits im Remote Repository, also zwischen Issue und Lieferung. Der Spieler der Release-Karte erhält zusätzliche Punkte wie auf der Issue-Karte angegeben.

Nach der Auswertung werden alle ausgelegten Karten und alle Handkarten abgeworfen. Jetzt können die Karten neu gemischt werden. Danach wird eine neue Issue-Karte ausgelegt und jeder Spieler erhält wieder drei Handkarten. Der Spieler nach dem Spieler, der die Release-Karte ausgespielt hat, beginnt die neue Runde.


\subsubsection*{Beispiel für eine Auswertung}
Am Ende einer Spielrunde kann das Remote Repository (also die ausgelegten Karten) folgendermaßen aussehen:

\begin{center}
\scalebox{0.5}{
\CardIssue{12}
\hskip1em
\CardFrontCommit{5}
\hskip-0.4\cardwidth
\clap{\raisebox{-0.05\cardwidth}{\PlayerMarkerWithShadow{errorcolor}{A}}}
\hskip0.4\cardwidth
\hskip1em
\cardnum=0
\CardFrontCommit{7}
\hskip-0.8\cardwidth
\raisebox{-0.05\cardwidth}{\CardMergeConflict{3}}
\hskip-0.65\cardwidth
\clap{\raisebox{0.05\cardwidth}{\PlayerMarkerWithShadow{nullcolor}{C}}}
\hskip0.65\cardwidth
\hskip1em
\cardnum=0
\CardRelease
}
\end{center}

Spieler~A hat die Release-Karte ausgespielt.
\begin{itemize}
	\item Spieler~A erhält 11~Punkte: Zahlkarte~5 plus 6~Extrapunkte fürs Release.
	\item Spieler~B erhält 0~Punkte: Seine Zahlkarte~7 wurde von Spieler~C geklaut!
	\item Spieler~C erhält 10~Punkte: Zahlkarte~7 und Zahlkarte~3 (Merge Conflict).
\end{itemize}

Die Punkte sollten in einer kleinen Tabelle aufgeschrieben werden:

\begin{center}
\begin{tabular}{c|c|c|c}
	Runde & A & B & C \\
	\hline
	v.\,1 & 11 & 0 & 10 \\
	\hline
	v.\,2 &  & & \\
	\hline
	v.\,3 &  & & \\
\end{tabular}
\end{center}



\section{Ende\label{ende}}
Am Anfang sollten alle Teilnehmer festlegen, wann das Spiel beendet ist (Definition of Done), z.\,B. nach 5~Lieferungen oder sobald ein Spieler mindestens 100~Punkte erreicht hat.

Gewonnen hat in jedem Fall selbstverständlich der Spieler mit der höchsten Punktezahl.



\clearpage
\section{Druckvorlagen}
Diese und die folgenden Seiten sind zum Ausdrucken und Ausschneiden vorgesehen. Zum Drucken wird dickes Papier empfohlen, z.\,B. 160\,g-Papier. Die Karten sollten natürlich beidseitig ausgedruckt werden. Die Ränder links und rechts sind gleich breit, so dass Vorder- und Rückseite immer aufeinander treffen.

Statt der Spielermarker auf dieser Seite können auch irgendwelche Spielfiguren oder sonstige Chips, Münzen o.\,ä.\ zum markieren der ausgelegten Karten verwendet werden.

\subsection*{Spielermarker}
\PlayerMarker{errorcolor}{A}
\PlayerMarker{errorcolor}{A}
\PlayerMarker{errorcolor}{A}
\PlayerMarker{errorcolor}{A}
\PlayerMarker{errorcolor}{A}
\PlayerMarker{errorcolor}{A}
\PlayerMarker{errorcolor}{A}

\PlayerMarker{warncolor}{B}
\PlayerMarker{warncolor}{B}
\PlayerMarker{warncolor}{B}
\PlayerMarker{warncolor}{B}
\PlayerMarker{warncolor}{B}
\PlayerMarker{warncolor}{B}
\PlayerMarker{warncolor}{B}

\PlayerMarker{nullcolor}{C}
\PlayerMarker{nullcolor}{C}
\PlayerMarker{nullcolor}{C}
\PlayerMarker{nullcolor}{C}
\PlayerMarker{nullcolor}{C}
\PlayerMarker{nullcolor}{C}
\PlayerMarker{nullcolor}{C}

\PlayerMarker{middlecolor}{D}
\PlayerMarker{middlecolor}{D}
\PlayerMarker{middlecolor}{D}
\PlayerMarker{middlecolor}{D}
\PlayerMarker{middlecolor}{D}
\PlayerMarker{middlecolor}{D}
\PlayerMarker{middlecolor}{D}

\PlayerMarker{highcolor}{E}
\PlayerMarker{highcolor}{E}
\PlayerMarker{highcolor}{E}
\PlayerMarker{highcolor}{E}
\PlayerMarker{highcolor}{E}
\PlayerMarker{highcolor}{E}
\PlayerMarker{highcolor}{E}

\PlayerMarker{oddcolor}{F}
\PlayerMarker{oddcolor}{F}
\PlayerMarker{oddcolor}{F}
\PlayerMarker{oddcolor}{F}
\PlayerMarker{oddcolor}{F}
\PlayerMarker{oddcolor}{F}
\PlayerMarker{oddcolor}{F}



\auxdim=3\cardwidth
\advance\auxdim by 3\cardlinewidth
\newgeometry{top=7mm,bottom=7mm,textwidth=\auxdim}
\Schnittmarken%
\cardnum=0
%
\CardFrontCommit{1}%
\CardFrontCommit{2}%
\CardFrontCommit{3}%
\CardFrontCommit{4}%
\CardFrontCommit{5}%
\CardFrontCommit{6}%
\CardFrontCommit{7}%
\CardFrontCommit{8}%
\CardFrontCommit{9}%

%\newgeometry{top=10mm,bottom=10mm,textwidth=3\cardwidth}
%
\Schnittmarken
\CardBack%
\CardBack%
\CardBack%
\CardBack%
\CardBack%
\CardBack%
\CardBack%
\CardBack%
\CardBack%



\Schnittmarken
\CardFrontCommit{10}%
\CardFrontCommit{12}%
\CardFrontCommitPlus[refactor]{%
\centerline{\rede{git push -- refactoring!}}\vskip1ex
	\Item Mit diesem Commit übernimmst du die vorherigen drei Commits als deine eigenen. Schieb sie unter diese Karte, alle diese Commits gelten ab jetzt als einer.}{0}{!}{errorcolor}%
\CardMergeConflict{2}%
\CardMergeConflict{3}%
\CardMergeConflict{4}%
\CardMergeRequest{2}%
\CardMergeRequest{3}%
\CardMergeRequest{4}%

\Schnittmarken
\CardBack%
\CardBack%
\CardBack%
\CardBack%
\CardBack%
\CardBack%
\CardBack%
\CardBack%
\CardBack%



\Schnittmarken
\CardRevert%
\CardRevert%
\CardRevert%
\CardFrontSign[??]{help}{???}{bg-green}{%
	\centerline{\rede{git help!}}\vskip1ex
	Du weißt gerade nicht weiter und brauchst Hilfe:\vskip0.5ex
	\Item Leg diese Karte auf den Abwurfstapel.\\
	\Item Zieh drei Handkarten zusätzlich, nachdem du \textrule{aufgestockt} hast.}%
\CardFrontSign{status}{?}{bg-yellow}{%
	\centerline{\rede{git status!}}\vskip1ex
	Du hast gerade keine Idee und brauchst erstmal einen Überblick:\vskip0.5ex
	\Item Leg diese Karte auf den Abwurfstapel.\\
	\Item Zieh eine Handkarte zusätzlich, nachdem du \textrule{aufgestockt} hast.}%
\CardFrontTikzBox{stash}{%
	% Mitte
	\begin{scope}[line width=2pt]
		\draw[fill=gray] (UCenter)+(-0.15\cardwidth,0) rectangle +(0.15\cardwidth,-0.075\cardheight);
		\draw[-latex, line width=3pt] (UCenter)+(0,0.1\cardheight)--(UCenter);
	\end{scope}
	% Ecken
	\DrawSignAtBorders{oddcolor}{%
		\draw[fill=gray, line width=1pt] (0,0)+(-0.05\cardwidth,0) rectangle +(0.05\cardwidth,-0.025\cardheight);
		\fill (0,0)--++(0.02\cardwidth,0.03\cardheight)--++(-0.04\cardwidth,0);
	}
	}{oddcolor}{%
	\centerline{\rede{git stash!}}\vskip1ex
	\Item Leg bis zu zwei Handkarten verdeckt vor Dir ab.
	\parendcentered
	\Item Sie können weiterhin wie Handkarten ausgespielt werden, zählen aber nicht beim \textrule{Aufstocken} mit und gehen bei einem Release nicht verloren.}%
\CardCherryPick%
%\CardFrontTikzBox{}{}{white}{todo}%
%\CardFrontTikzBox{}{}{white}{todo}%
\CardMergeRequest{5}
\CardRevert

\Schnittmarken
\CardBack%
\CardBack%
\CardBack%
\CardBack%
\CardBack%
\CardBack%
\CardBack%
\CardBack%
\CardBack%



\Schnittmarken
\CardRelease%
\CardRelease%
\CardRelease%
\CardRelease%
\CardRelease%
\CardBug{5}%
\CardBug{7}%
\CardBug{8}%
\CardBug[0.275\cardwidth]{10}%

\Schnittmarken
\CardBack%
\CardBack%
\CardBack%
\CardBack%
\CardBack%
\CardBack%
\CardBack%
\CardBack%
\CardBack%



\Schnittmarken
\CardIssue{10}%
\CardIssue{12}%
\CardIssue{15}%
\CardIssue{16}%
\CardIssue{20}%
\CardIssue{24}%
\CardIssue{26}%
\CardIssue{30}%
\CardIssue{32}%

\Schnittmarken
\CardBack[bgcolor2]%
\CardBack[bgcolor2]%
\CardBack[bgcolor2]%
\CardBack[bgcolor2]%
\CardBack[bgcolor2]%
\CardBack[bgcolor2]%
\CardBack[bgcolor2]%[bgcolor3]%
\CardBack[bgcolor2]%[bgcolor3]%
\CardBack[bgcolor2]%[bgcolor3]%



% Zusatzkarten
\Schnittmarken
\CardFrontCommit{3}%
\CardFrontCommit{4}%
\CardFrontCommit{5}%
\CardFrontCommit{6}%
\CardFrontCommit{7}%
\CardFrontCommit{8}%
\CardMergeConflict{1}%
\CardRevert%
\CardCherryPick%

\Schnittmarken
\CardBack%
\CardBack%
\CardBack%
\CardBack%
\CardBack%
\CardBack%
\CardBack%
\CardBack%
\CardBack%

\end{document}
