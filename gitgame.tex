\documentclass[DIV=12]{scrartcl}
\setkomafont{sectioning}{\normalfont\bfseries\boldmath}

\usepackage{geometry}

\usepackage[british, main=ngerman]{babel}
\usepackage{tikz}

\setlength\parindent{0pt}
\setlength\parfillskip{1em plus 1fil}
\setlength\parskip{0pt}

%\usepackage{fontspec}
%\setmainfont{Linux Libertine O}
%\setsansfont{Linux Biolinum O}
\newcommand\hugefont[1]{{\fontsize{50}{60}\sffamily\bfseries\selectfont #1}}
\newcommand\bigfont[1]{{\fontsize{20}{24}\sffamily\bfseries\selectfont #1}}
%\newcommand\textfontsize[1]{\fontsize{10}{12}\sffamily\selectfont #1}
\newcommand\textfontsize[1]{\fontsize{9}{10}\sffamily\selectfont #1}

\newcommand\rede[1]{\glqq #1\grqq{}}
\newcommand\Item{$\bullet$\kern0.333em}

\newdimen\cardwidth
\cardwidth=60mm
\newdimen\cardheight
\cardheight=90mm
\newdimen\cardborder
\cardborder=5mm
\newdimen\cardtextwidth
\cardtextwidth=\cardwidth
\advance\cardtextwidth by -2.5\cardborder
\newdimen\cardlinewidth
\cardlinewidth=0.4pt

\newcount\auxnum
\newcommand\CardBack{%
  \begin{tikzpicture}
	\draw[line width=\cardlinewidth, rounded corners=5mm] (0,0) rectangle (\cardwidth,\cardheight) coordinate (URex);
	\path (0,0)--(\cardwidth,\cardheight) coordinate[midway] (Center);
	\path (URex)--+(-\cardborder,-\cardborder) coordinate (UR);
	\fill[green!20, rounded corners=\cardborder] (\cardborder,\cardborder) coordinate (LL) rectangle (UR);
    \node at (Center) {\hugefont{git}};
  \end{tikzpicture}%
}
\newenvironment{CardFront}[1]{%
  \begin{tikzpicture}
    \draw[line width=\cardlinewidth, rounded corners=5mm] (0,0) rectangle (\cardwidth,\cardheight) coordinate (URex);
    \path (0,0)--(\cardwidth,\cardheight) coordinate[midway] (Center);
    \path (URex)--+(-\cardborder,-\cardborder) coordinate (UR);
    \fill[gray!20, rounded corners=\cardborder] (\cardborder,\cardborder) coordinate (LL) rectangle (UR);
    \path (Center)+(0,0.05\cardheight) coordinate (UCenter);
	
	% Überschrift
	\path (UCenter)+(0,0.25\cardheight) node {\bigfont{#1}};
}{
  \end{tikzpicture}%
}
\colorlet{errorcolor}{red!75}
\colorlet{warncolor}{orange!75}
\colorlet{nullcolor}{yellow!75}
\colorlet{middlecolor}{green!65}
\colorlet{highcolor}{blue!65}
\newcommand\CardTextField[1]{%
    \path (UCenter)+(0,-0.25\cardwidth-0.025\cardheight) node[anchor=north] {%
	\begin{minipage}{\cardtextwidth}
		\textfontsize\rightskip0pt plus 1em
		#1
	\end{minipage}};
}
\newcommand\CardFrontNumbered[3][commit]{% Überschrift, Zahl, Text
  \begin{CardFront}{#1}
    \ifnum#2<8
      \auxnum=#2
      \multiply\auxnum by 12
      \colorlet{numcolor}{middlecolor!\the\auxnum!nullcolor}
      \def\eight{white}
      \auxnum=#2
    \else
      \auxnum=#2
      \advance\auxnum by -8
      \multiply\auxnum by 12
      \colorlet{numcolor}{highcolor!\the\auxnum!middlecolor}
      \def\eight{black}
      \auxnum=#2
      \advance\auxnum by -8
    \fi
    % Zahlwert
    \draw[line width=1.5pt, fill=numcolor] (UCenter) circle[radius=0.25*\cardwidth];
    \path (UCenter)+(0,0.025\cardwidth) node[anchor=center] {\hugefont{#2}};

    % Textbereich
    \CardTextField{#3}
    
    \ifnum\auxnum<4
    	\def\four{white}
    \else
    	\def\four{black}
    	\advance\auxnum by -4
    \fi
    \ifnum\auxnum<2
		\def\two{white}
	\else
		\def\two{black}
		\advance\auxnum by -2
	\fi
    \ifnum\auxnum<1
		\def\one{white}
	\else
		\def\one{black}
	\fi
    \path (UCenter)++(-0.09*\cardwidth,-0.125*\cardwidth) coordinate (aux);
    \draw[fill=\eight] (aux) circle[radius=1mm];
    \path (aux)++(0.06*\cardwidth,0) coordinate (aux);
	\draw[fill=\four] (aux) circle[radius=1mm];
    \path (aux)++(0.06*\cardwidth,0) coordinate (aux);
	\draw[fill=\two] (aux) circle[radius=1mm];
    \path (aux)++(0.06*\cardwidth,0) coordinate (aux);
	\draw[fill=\one] (aux) circle[radius=1mm];
  \end{CardFront}%
}
\newcommand\CardFrontCommit[2][\centering\rede{git push!}]{\CardFrontNumbered{#2}{#1}}% Text, Zahl
\newcommand\CardFrontSign[4]{% Überschrift, Zeichen, Farbe, Text
  \begin{CardFront}{#1}
	% Zeichen
	\draw[line width=1.5pt, fill=#3] (UCenter) circle[radius=0.25*\cardwidth];
	\path (UCenter) node[anchor=center] {\hugefont{#2}};

	% Textbereich
	\CardTextField{#4}
  \end{CardFront}%
}
\newcommand\CardFrontTikz[4]{% Überschrift, Tikz-Graphik, Farbe, Text
  \begin{CardFront}{#1}
	% Tikz-Graphik
	\draw[line width=1.5pt, fill=#3] (UCenter) circle[radius=0.25*\cardwidth];
	#2
	
	% Textbereich
	\CardTextField{#4}
  \end{CardFront}%
}

\begin{document}
Ein Test für Sonderzeichen: ÄÖÜ äöü ß.

\vfill
Es folgen die Seiten mit den Karten zum Ausdrucken.
\newdimen\auxdim
\auxdim=3\cardwidth
\advance\auxdim by 3\cardlinewidth,
\newgeometry{top=10mm,bottom=10mm,textwidth=\auxdim}

\CardFrontSign{help}{?}{warncolor}{%
	\rede{Ich weiß gerade nicht weiter und brauche Hilfe: git help!}\\[1ex]
	\Item Leg diese Karte auf den Abwurfstapel.\\
	\Item Zieh drei neue Handkarten.}%
\CardFrontSign{status}{?}{nullcolor}{%
	\rede{Ich habe gerade keine Ideen und muss mir erstmal einen Überblick verschaffen: git status!}\\[1ex]
	\Item Leg diese Karte auf den Abwurfstapel.\\
	\Item Zieh eine neue Handkarte.}%
\CardFrontNumbered{3}{%
	\centerline{\textbf{\MakeUppercase{Merge Conflict}}}
	\rede{git push -- Dabei verursache und löse ich einen Merge Conflict!}\\[1ex]
	\Item Leg den vorherigen Commit als Deinen eigenen an, unter diese Karte.}%

\CardFrontTikz{revert}{%
	\begin{scope}[line width=2pt]
		\path (UCenter)--+(0.05\cardwidth,0.075\cardheight) coordinate (aux);
		\node[draw, fill=gray, circle, radius=1.5mm]  (A) at (aux) {};
		\path (UCenter)--+(0.05\cardwidth,-0.075\cardheight) coordinate (aux);
		\node[draw, fill=gray, circle, radius=1.5mm]  (B) at (aux) {};
		\draw (A)--(B);
		\draw[->] (A)..controls +(-0.2\cardwidth,0) and +(-0.2\cardwidth,0)..(B);
	\end{scope}
	}{errorcolor}{%
	\rede{Der gerade geschriebene Code macht schwere Fehler und muss wieder weg: git revert!}\\[0.5ex]
	\Item Muss unmittelbar auf eine Zahlkarte folgen und negiert sie.\\
	\Item Leg beide Karten auf den Abwurfstapel.}%
\CardFrontTikz{stash}{%
	\begin{scope}[line width=2pt]
		\draw[fill=gray] (UCenter)+(-0.15\cardwidth,0) rectangle +(0.15\cardwidth,-0.075\cardheight);
		\draw[->, line width=3pt] (UCenter)+(0,0.1\cardheight)--(UCenter);
	\end{scope}
	}{blue!50}{%
	\centerline{\rede{git stash!}}%\\[1ex]
	\vskip1ex
	\Item Leg drei Handkarten vor Dir ab.\\
	\Item Diese können weiterhin wie Handkarten ausgespielt werden, gehen aber bei einem Release nicht verloren.}%
\CardFrontTikz{release}{%
	}{middlecolor}{\rede{Ich beschließe, dass der aktuelle Stand ausgeliefert wird: -- git tag!}\\[0.5ex]
	\Item Werte alle Karten im Repo.\ aus.\\
	\Item Leg die Karten vom Repo., die aktuellen Issues und dieser Karte auf den Abwurfstapel.\\
	\Item Ziehe dann einen neuen CR.}%

\clearpage
\CardBack%
\CardBack%
\CardBack%

\CardBack%
\CardBack%
\CardBack%

\CardBack%
\CardBack%
\CardBack%


\CardFrontCommit{1}%
\CardFrontCommit{2}%
\CardFrontCommit{3}%

\CardFrontCommit{4}%
\CardFrontCommit{6}%
\CardFrontCommit{8}%

\CardFrontCommit{10}%
\CardFrontCommit{12}%
\CardFrontCommit{15}%

%\newgeometry{top=10mm,bottom=10mm,textwidth=3\cardwidth}
\CardBack%
\CardBack%
\CardBack%

\CardBack%
\CardBack%
\CardBack%

\CardBack%
\CardBack%
\CardBack%

\end{document}
