% !TeX spellcheck = de_DE
\documentclass[DIV=15, fontsize=11pt]{gitgame}
\renewcommand*{\sectionformat}{\thesection~--~}
\renewcommand*{\subsectionformat}{\thesubsection~--~}

\usepackage[top=3mm,bottom=3mm,textwidth=\cardwidth,textheight=\cardheight,paperwidth=65mm,paperheight=98mm]{geometry}

\usepackage[provide=*, british, main=ngerman]{babel}
\usepackage[hidelinks]{hyperref}



\begin{document}
\cardnum=0
\showouterborderfalse
%
\CardFrontCommitPlus[refactoring]{%
	\centerline{\rede{git push -- refactoring!}}\vskip1ex
	\Item Mit diesem Commit übernimmst du die vorherigen drei Commits als deine eigenen. Schieb sie unter diese Karte, alle diese Commits gelten ab jetzt als einer.}{0}{!}{errorcolor}%

\CardFrontCommit{1}%

\CardMergeConflict{1}%

\CardFrontCommit{2}%

\CardMergeConflict{2}%

\CardMergeRequest{2}%

\CardFrontCommit{3}%

\CardFrontCommit{3}%

\CardMergeConflict{3}%

\CardMergeRequest{3}%

\CardFrontCommit{4}%

\CardFrontCommit{4}%

\CardMergeConflict{4}%

\CardMergeRequest{4}%

\CardFrontCommit{5}%

\CardFrontCommit{5}%

\CardMergeRequest{5}%

\CardFrontCommit{6}%

\CardFrontCommit{6}%

\CardFrontCommit{7}%

\CardFrontCommit{7}%

\CardFrontCommit{8}%

\CardFrontCommit{8}%

\CardFrontCommit{9}%

\CardFrontCommit{10}%

\CardFrontCommit{12}%


\CardRelease%

\CardRelease%

\CardRelease%

\CardRelease%

\CardRelease%


\CardRevert%

\CardRevert%

\CardRevert%

\CardRevert%

\CardRevert%


\CardFrontSign[??]{help}{???}{bg-green}{%
	\centerline{\rede{git help!}}\vskip1ex
	Du weißt gerade nicht weiter und brauchst Hilfe:\vskip0.5ex
	\Item Leg diese Karte auf den Abwurfstapel.\\
	\Item Zieh drei Handkarten zusätzlich, nachdem du \textrule{aufgestockt} hast.}%

\CardFrontSign{status}{?}{bg-yellow}{%
	\centerline{\rede{git status!}}\vskip1ex
	Du hast gerade keine Idee und brauchst erstmal einen Überblick:\vskip0.5ex
	\Item Leg diese Karte auf den Abwurfstapel.\\
	\Item Zieh eine Handkarte zusätzlich, nachdem du \textrule{aufgestockt} hast.}%

\CardFrontTikzBox{stash}{%
	% Mitte
	\begin{scope}[line width=2pt]
		\draw[fill=gray] (UCenter)+(-0.15\cardwidth,0) rectangle +(0.15\cardwidth,-0.075\cardheight);
		\draw[-latex, line width=3pt] (UCenter)+(0,0.1\cardheight)--(UCenter);
	\end{scope}
	% Ecken
	\DrawSignAtBorders{oddcolor}{%
		\draw[fill=gray, line width=1pt] (0,0)+(-0.05\cardwidth,0) rectangle +(0.05\cardwidth,-0.025\cardheight);
		\fill (0,0)--++(0.02\cardwidth,0.03\cardheight)--++(-0.04\cardwidth,0);
	}
	}{oddcolor}{%
	\centerline{\rede{git stash!}}\vskip1ex
	\Item Leg bis zu zwei Handkarten verdeckt vor Dir ab.
	\parendcentered
	\Item Sie können weiterhin wie Handkarten ausgespielt werden, zählen aber nicht beim \textrule{Aufstocken} mit und gehen bei einem Release nicht verloren.}%

\CardCherryPick%

\CardCherryPick%


\CardBug{5}%

\CardBug{7}%

\CardBug{8}%

\CardBug[0.275\cardwidth]{10}%



\CardIssue{10}%

\CardIssue{12}%

\CardIssue{15}%

\CardIssue{16}%

\CardIssue{20}%

\CardIssue{24}%

\CardIssue{26}%

\CardIssue{30}%

\CardIssue{32}%


\end{document}
